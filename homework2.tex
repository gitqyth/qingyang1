\documentclass[a4paper]{ctexart}
\usepackage{xltxtra} 


\title{学习报告}

\author{唐浩 \\ 信息与计算科学 3200102118}

\begin{document}

\maketitle

\pagestyle{empty}

\section{shell文件运行}

\begin{itemize}
\item shell脚本文件内容:
\end{itemize}

\begin{verbatim}
#!/bin/sh

salutation=”Hello”
echo $salutation
echo “The program $0 is now running”
echo “The second parameter was $2”
echo “The first parameter was $1”
echo “The parameter list was $*“
echo “The user’s home directory is $HOME”
echo “Please enter a new greeting”
read salutation
echo $salutation
echo “The script is now complete”
exit 0
\end{verbatim}

\begin{itemize}
  
\item 尝试运行:
  
\end{itemize}

\begin{verbatim}
sh try.sh
\end{verbatim}

\begin{itemize}
  
\item 运行结果如下:
  
\end{itemize}

\begin{verbatim}
Hello
The program ./try_var is now running
The second parameter was bar
The first parameter was foo
The parameter list was foo bar baz
The user’s home directory is /home/rick
Please enter a new greeting
Sire
Sire
The script is now complete
\end{verbatim}


\begin{itemize}
\item 工作原理:
\end{itemize}

\begin{verbatim}
操纵参数和环境变量;
该脚本创建变量salutation,显示其内容;
然后显示各种参数变量和环境变量$HOME是如何存在并具有适当值的。
\end{verbatim}

\section{个人理解}

对于 \verb\#!/bin/sh,虽然shell编程是以 \# 为注释,但是 "\verb\#!/bin/sh" 却不是。它是对shell的声明,说明你所用的是那种类型的shell及其路径所在。
(它是指此脚本使用/bin/sh来解释执行,\verb\#!是特殊的表示符,其后面跟的是解释此脚本的shell的路径)
如果没有声明,则脚本将在默认的shell中执行,默认shell是由用户所在的系统定义为执行shell脚本的shell.如果脚本被编写为在Kornshell ksh中运行,而默认运行shell脚本的为C shell csh,则脚本在执行过程中很可能失败。所以建议大家就把 "\verb\#!/bin/sh" 当成C 语言的main函数一样,写shell必须有,以使shell程序更严密;

echo用于显示字符串,直接在echo后面加上想要显示的内容就好。在echo的后面,不仅可以加字符串,还可以加变量名:

\begin{verbatim}
#定义变量str
str = "Hello world"

#在echo后加上str变量,一样可以显示出来
echr "$str, good morning"

Hello world, good morning
\end{verbatim}




\subsection{管道和重定向}
比如: \verb|ls -l| 是列出当前目录下的全部文件, 但这样会导致一屏幕可能放不下.
\verb|more| 命令则是将一个超出一屏幕的输出, 在输出一满屏内容后暂停, 然后可以逐行(按回车)输出.
将两个命令用 \verb!|! 管道(pipe) 链接在一起: \verb!ls -l | more! 就是将当前目录下所有文件列出,
按屏停顿.

我们之前还有一个例子, \verb|env| 实际上是输出全部环境变量, 而 \verb|grep| 是过滤的意思,
所以 \verb!env | grep PATH! 就是输出全部环境变量中包含 \verb|PATH| 内容的东西.

这些组合都可以灵活搭配使用. UNIX 哲学在所有分支中都是通行的. 

大家可能注意到, shell 本身也是一个命令. 事实上, 目前流行的 shell 有多个版本, 我们使用的是 bash,
这也是开源社区最通行的版本. 使用 Mac OS 的同学可能注意到 Mac 的默认 shell 不是 bash.
这个命令本身的位置在 \verb|/bin/bash|. Linux 的命令实际上在 shell 中运行和反馈, shell 则提供 pipe
和环境变量等等这样的运行环境. 除了 pipe, shell 还有一个重要特性是重定向(redirection).

我们知道, C 语言中输出, 用 \verb|printf| 函数, 本质上是向一个被称为 \verb|stdout|
的设备文件输出字符流. \verb|printf| 和 \verb|stdout| 都在头文件 \verb|stdio.h| 
定义. 在正常情况下, \verb|stdout| 是定义在显示器上的. 但我们可以用重定向, 将其重定向为一个文本文件.
比如:
\begin{verbatim}
ls -l > re.txt
\end{verbatim}
将当前目录的内容重定向到 \verb|re.txt| 下. 这里 \verb|>| 每一次都会重新生成一个新的输出结果文件.
而 \verb|>>| 则是在文件结尾新增.
\begin{verbatim}
ls -l >> re.txt
\end{verbatim}

有的时候, 我们写了一个程序, 会有大量的输出, 如果从屏幕走, 记录本身就很麻烦.
这时你没有必要专门做一套文本文件的读写机制, 而只需要重定向就行了. 除了 stdout 可以重定向,
stderr 也是可以重定向的. 为了区别, 我们可以分别用 1(可以忽略) 和 2 表示这两个不同的通道. 比如:
\begin{verbatim}
./count 10 >std.txt 2>err.txt
\end{verbatim}
或者等价地
\begin{verbatim}
./count 10 1>std.txt 2>err.txt
\end{verbatim}
注意这里 \verb|1>| 和 \verb|2>| 之间不要有空格, 否则会和命令行参数混淆.

这几件规则可以结合在一起发挥作用. 比如:
\begin{verbatim}
./count 10 | grep count
\end{verbatim}
这里你仔细观察, 就会发现, 管道只接管了 \verb|std|, 而且管道之后的命令的执行和
\verb|count| 是同步的. 两边的输出甚至可能交错. 这里涉及到进程并行的问题. 我们先不讨论.
所以我们可以先清理一下输出, 比如把 \verb|stderr| 先重定向掉. 
\begin{verbatim}
./count 10 2>/dev/null | grep count
\end{verbatim}

\subsection{shell 编程}

我们可以做的更过份一点. 比如我们知道在 \verb|/usr/include| 下放了很多 C 语言的头文件,
比如 stdio.h 就在这里. 现在我们想看一下这些头文件中有多少包含了 \verb|stdout| 这个关键字.
比如我们想看一下这个文件是在哪个文件定义的. 那么我们可以用命令:

\begin{verbatim}
grep stdout /usr/include/*
\end{verbatim}

找到这些文件, 然后记下来, 再一个个文件去读. 这就有点蠢了. 就不能一次搞定? 找到这些包含
\verb|stdout| 的头文件, 找到一个, 读一个, 这样就很快能发现是谁定义, 还能马上看到定义周围的细节.
要做到这件事, 我们可以写一个段小的脚本:

\begin{verbatim}
$ for file in /usr/include/*
> do
> if grep -l stdio $file
> then
> more $file
> fi
> done
\end{verbatim}

这样就可以愉快地做到了. 注意我们可以用上箭头调用上一次运行的命令, 用 TAB 键自动补齐一个最有可能的命令.
我们这里用一下上箭头, 你会看到原来刚才那一长串都算一个命令.

考虑到这个命令输入太麻烦, 我们不如把它做成一个可执行脚本算了.
查看脚本 \verb|first|. 





\bibliographystyle{plain}
\bibliography{crazyfish.bib}

\end{document}
