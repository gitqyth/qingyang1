\documentclass{ctexart}

\usepackage{graphicx}
\usepackage{amsmath}

\title{作业一:罗必达问题的叙述与证明}


\author{作者姓名 : 唐浩 \\ 作者专业和学号 : 信息与计算科学 3200102118}

\begin{document}

\maketitle


洛必达法则是在一定条件下通过分子分母分别求导再求极限来确定未定式值的方法。众所周知,两个无穷小之比或两个无穷大之比的极限可能存在,也可能不存在。因此,求这类极限时往往需要适当的变形,转化成可利用极限运算法则或重要极限的形式进行计算。洛必达法则便是应用于这类极限计算的通用方法。
\section{问题描述}
问题叙述如下: 洛必达法则的本质是一个定理,它规定,如果一个形如
的极限,如果它满足:

1. x趋向于常数a时,函数f(x)和F(x)都趋向于0;

2.在点a的去心邻域内,f(x)和F(x)的导数都存在,并且{F'(x) {\ne 0}};

3.如果

\[
\lim_{x \rightarrow 0}
\frac{f(x)}{F(x)}
\]

存在,那么:

\[
\lim_{x \rightarrow a}
\frac{f(x)}{F(x)} =
\lim_{x \rightarrow a}
\frac{f'(x)}{F'(x)}
\]

也就是当变量趋向于一个常数时,如果分子分母函数的导数存在,那么我们可以用导数的极限比值来代替原函数的比值。

\section{证明}
由于函数在a点的去心邻域可导,也就是说函数在这个a的去心邻域内连续。那么我们套用柯西中值定理,在x趋向于a时,可以得到在区间(a, x)内找到一个点 \xi

使得:

\[
\frac{f(x) - f(a)}{F(x) - F(a)} =
\frac{f'(\xi)}{F'(\xi)}
\]

到这里还差一点,因为还少了一个条件,书上的解释是由于函数比值的极限与函数值无关,所以可以假设f(a)和F(a)等于0。其实我们只要将这两做差,证明一下差值等于0即可。将

\[
\lim_{x \rightarrow a}
\frac{f(x) - f(a)}{F(x) - F(a)} -
\frac{f(x)}{F(x)}
\]

通分之后,可以得到:

\[
\lim_{x \rightarrow a} f(x)F(a) - f(a)F(x)
\]

到这里,不难看出来,当x趋向于a的时候,上面的差值趋向于0,所以:
\[
\frac{f(x) - f(a)}{F(x) - F(a)} =
\frac{f'(\xi)}{F'(\xi)} =
\frac{f(x)}{F(x)}
\]

由于x趋向于a的时候,\xi也趋向于a,那么我们就得到了:

\[
\lim_{x \rightarrow a}
\frac{f(x)}{F(x)} =
\lim_{x \rightarrow a}
\frac{f'(x)}{F'(x)}
\]


\end{document}
