\documentclass[a4paper,12pt]{ctexart}
\usepackage{float}
\usepackage{xltxtra} 
\newcommand{\tabincell}[2]{\begin{tabular}{@{}#1@{}}#2\end{tabular}}

\title{学习报告}

\author{唐浩 \\ 信息与计算科学 3200102118}

\begin{document}

\maketitle

\pagestyle{empty}

\section{shell文件运行}

\begin{itemize}
\item shell脚本文件内容:
\end{itemize}

\begin{verbatim}
#!/bin/sh

salutation = "Hello”
echo $salutation
echo “The program $0 is now running”
echo “The second parameter was $2”
echo “The first parameter was $1”
echo “The parameter list was $*"
echo “The user’s home directory is $HOME”
echo “Please enter a new greeting”
read salutation
echo $salutation
echo “The script is now complete”
exit 0
\end{verbatim}

\begin{itemize}
\item 尝试运行:
\end{itemize}

\begin{verbatim}
sh try.sh
\end{verbatim}

\begin{itemize}
\item 运行结果如下:
\end{itemize}

\begin{verbatim}
Hello
The program ./try_var is now running
The second parameter was bar
The first parameter was foo
The parameter list was foo bar baz
The user’s home directory is /home/rick
Please enter a new greeting
Sire
Sire
The script is now complete
\end{verbatim}


\begin{itemize}
\item 工作原理:
\end{itemize}

\begin{verbatim}
操纵参数和环境变量;
该脚本创建变量salutation,显示其内容;
然后显示各种参数变量和环境变量$HOME是如何存在并具有适当值的。
\end{verbatim}

\section{个人理解}

对于 \verb\#!/bin/sh,虽然shell编程是以 \verb\# 为注释,但是它却不是。
\verb\#!/bin/sh 是对shell的声明;

(指此脚本使用/bin/sh(shell脚本路径)来解释执行,\verb\#!是特殊的表示符)

如果没有声明,则脚本将在默认的shell中执行,默认shell是由用户所在的系统定义为执行shell脚本的shell.如果脚本被编写为在Kornshell ksh中运行,而默认运行shell脚本的为C shell csh,则脚本在执行过程中很可能失败,所以建议大家把 "\verb\#!/bin/sh" 当成C 语言的main函数一样。

echo用于显示字符串,直接在echo后面加上想要显示的内容就好。在echo的后面,不仅可以加字符串,还可以加变量名:
\begin{verbatim}
#定义变量str
str = "Hello world"
#在echo后加上str变量,一样可以显示出来
echr ``$str, good morning"

Hello world, good morning
\end{verbatim}

echo后面接不同的引号也会有不同的效果:

单引号将所有字符都看成普通字符,双引号会解释\$ , \textbackslash 和  ' 这三种特殊字符,不加引号的话则会解释所有特殊字符。效果如下:

\begin{table}[H]
%\newcommand{\tabincell}[2]{\begin{tabular}{@{}#1@{}}#2\end{tabular}}
  \centerinng
\begin{tabular}{|c|c|c|}
  \hline
  输入命令 & \tabincell{c}{输出内容} & \tabincell{c}{解释} \\ \hline
  \tabincell{c}{echo `$USER* $(date)'} & \tabincell{c}{$USER*$(date)} & \tabincell{c}{单引号无视所有特殊字符\\ 所有字符在它眼里都是普通字符\\ 都是芸芸众生} \\ \hline
  \tabincell{c}{echo ``$USER* $(date)"} & \tabincell{c}{qingyang*\\ 2022年 06月 28日\\ 星期二 19:49:01 CST} & \tabincell{c}{双引号会无视文件通配符\\ 但是``\$" ``\textbackslash" 会起作用} \\ \hline
  
\end{tabular}
\end{table}

\bibliographystyle{plain}
\bibliography{crazyfish.bib}

\end{document}
