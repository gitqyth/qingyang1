\documentclass{ctexart}

\usepackage{graphicx}
\usepackage{amsmath}
\usepackage{xltxtra}
\usepackage[T1]{fontenc}

\title{我的 Linux 工作环境}

\author{作者姓名: 唐浩 \\ 作者专业和学号: 信息与计算科学 3200102118}

\begin{document}

\maketitle

以下我将简要介绍我自己的 Linux 工作环境:

\section{发行版本名称及版本号}

版本名称: ubuntu 20.04.4 LTS

版本号: 20.04

\section{调整系统}

\subsection{系统安装}

我在网上搜索教程,进行了一个双系统的安装(Windows 10 系统以及 Ubuntu 20.04 版本),由于我电脑里面转载了两块硬盘,故我不需要将 Windows 系统所占用的空间进行分割,我将两个系统分别放进了两个硬盘空间里面。(我采用的是U盘安装系统)\cite{orangeye2009玩转}

\subsection{系统配置}

安装完成 Ubuntu 系统后,我依照钉钉里面学在浙大课程 ``数学软件” 里面的课件——设置你的工作环境,进行 Linux 工作环境的设置{\bf (当然,在此之前我已经 install 了 synaptic)};跟着讲解视频,我依次标记了 gcc, g++, emacs, dxygen, make, cmake, automake, ssh, git, x11, dx 等许多安装包,标记完成后,点击 Apply ,我采用的是 cn99 的源,速度还行吧。此外,我还自主安装了 MATLAB 2022a 以及 Dingtalk ,正准备尝试安装 QQ 以及 WeChat, 当然还有 Visual Studio Code。因为需要打开 pdf 文件,我同时安装了 okular,倒是 doxymacs 装不了。我还为 emacs 设置了可以输入中文的环境,在 .bashrc 中相关位置输入:

{\bf alias emacs=`LC\_CTYPE=zh\_CN.utf8 emacs' }

\section{规划工作}

再下一步的话,我会将 Ubuntu 系统尽量配置成我熟悉的环境,开始熟练使用 Ubuntu,尽量减少由 Windows 带过来的一些不必要的习惯。我将继续学习使用 Latex, 学会使用命令来控制计算机,尽量多敲几遍 Latex 经常会用到的代码,并借此记住此代码,做到在没有模板的情况下,可以独自进行 Latex 文档的编辑输出。

\subsection{使用场合}

由于我决定从 Windows 转线到 Linux,所以今后但凡会用到电脑的地方,我都是使用 Ubuntu 系统进行操作,例如:

2022-2023学年,秋冬学期的{\bf 数据结构和算法}这门课程,这门课需要我们使用 Ubuntu 系统下的 emacs 进行编译,主要使用 C++ 语言,这便要求我们使用 Linux 环境。

2022-2023学年,秋冬学期的{\bf 数值分析}这门课程,这门课是由{\bf (老妖)}王何宇老师进行教授,依其的性格,多半也是使用 Linux 环境。

其他的场合,多半是自己练习,自己进行编码的训练的时候会用到,毕竟我已经不准备使用 Windows 系统了。

{\bf 此后但凡电脑所用之处,便是我使用 Linux 环境之时!}

\subsection{分析环境}

我觉得我目前的工作环境足够我现在所进行的一些工作,也就是满足了我当前的需求,至于未来,我认为是不够的,毕竟我如今的工作环境只是进行了一个初步的配置,还需要些什么配置,大概也只能等遇到问题的时候进行解决了,或者提前询问相关学长、老师,以获取一些建议。

\section{工作保证}

首先,你得保证自己的代码、文件等信息不会因为一些意外而导致丢失,这个时候,一些远程仓库就i很有必要了;你可以使用 github, gitee, 坚果云等,来存储自己代码文件,并且保证时常更新;

每天检查系统更新,日常进行dbg;

记住重要文件的位置,尽量不要因为忘记文件位置而最后导致文件的丢失;

日志对于安全来说非常重要。日志里面记录了系统每天发生的各种各样的事情。可以通过日志来检查系统发生错误的原因,或者系统遭受攻击时留下的痕迹。日志的功能主要有:审计和监测。日志也可以实时的监测系统状态,监测和追踪入侵者。\cite{李海涛2008基于}

\newpage

\bibliographystyle{plain}
\bibliography{ckwx.bib}


\end{document}
